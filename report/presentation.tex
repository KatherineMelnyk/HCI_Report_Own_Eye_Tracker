\documentclass{beamer}
\usepackage[utf8]{inputenc}
\usepackage[numbers]{natbib}
\usepackage{amsmath}
\usepackage{graphicx}
\usepackage{wrapfig}
\usepackage{lipsum}
\usepackage{siunitx}
\usepackage{xcolor}
\graphicspath{ {photo/} }

\usetheme{AnnArbor}
\usecolortheme{beaver}

%! Author = cyberstar
%! Date = 26.10.20
\title[Homemade Eye Tracker] %optional
{\textbf{Constructing a Homemade Eye Tracker from Consumer-Grade Web Cameras}}

\author[Sam, Kate] % (optional, for multiple authors)
{\textbf{Samantha Aziz}\inst{1} \and \textbf{Kateryna Melnyk}\inst{1}}

\institute[TXST] % (optional)
{
\inst{1}%
Faculty of Computer Science\\
Texas State University
}

\date[10/28/20] % (optional)
{CS 7321 — Human Computer Interaction}

\begin{document}

    \frame{\titlepage}

    \begin{frame}
        \frametitle{Hardware: LifeCam VX-1000}

        \center
        For our experiment we use two cameras.
        Firstly, we modified a \textbf{Microsoft LifeCam VX-1000} to enable eye
        tracking functionality.

        \begin{figure}
            \begin{center}
                \includegraphics[width=0.2\textwidth]{VX_1000.jpg}
            \end{center}
            \caption{LifeCam VX-1000}
            \label{fig:VX_100.}
        \end{figure}

    \end{frame}

    \begin{frame}
        \frametitle{Hardware: ToLuLu 1080p webcam}
        \center Secondly, we continue our work with a \textbf{ToLuLu 1080p
        webcam.}

        \begin{figure}
            \begin{center}
                \includegraphics[width=0.4\textwidth]{ToLuLu.jpg}
            \end{center}
            \caption{ToLuLu 1080p webcam.}
            \label{fig:Tolulu}
        \end{figure}

    \end{frame}

    \begin{frame}
        \frametitle{Modification of the camera to remove IR filter}
        \center First, we removed the \textbf{IR} filter from the disassembled camera.

        \begin{figure}
            \begin{center}
                \includegraphics[width=0.8\textwidth]{Red_filter.jpg}
            \end{center}
            \caption{Camera with red filter and without it.}
            \label{fig:red_filter}
        \end{figure}

    \end{frame}

    \begin{frame}
        \frametitle{Insertion of the visible light filter}
        \center
        The visible light filter - \textbf{"Kodak GC Ultramax 400 ASA 24 Exposure
        35mm Color Film"}.

        \begin{figure}
            \begin{center}
                \includegraphics[width=0.8\textwidth]{Visible_light.jpg}
            \end{center}
            \caption{Visible light filter.}
            \label{fig:VL_filter}
        \end{figure}

    \end{frame}

    \begin{frame}
        \frametitle{Insertion of the visible light filter}
        \center

        We then replaced the \textbf{IR}(infra-red) filter with a sheet of
        exposed, undeveloped camera film to filter natural light.

        \begin{figure}
            \begin{center}
                \includegraphics[width=0.8\textwidth]{VL_cameras.jpg}
            \end{center}
            \caption{Visible light filter in ToLuLu 1080p webcam and LifeCam
            VX-1000.}
            \label{fig:VL_cameras}
        \end{figure}

    \end{frame}

    \begin{frame}
        \frametitle{Assembly of the IR light}
        \center
        To improve tracking accuracy, we also attached a \textbf{1.5V infrared
        LED} to the top of the camera.

        \begin{figure}
            \begin{center}
                \includegraphics[width=0.8\textwidth]{IR_cameras.jpg}
            \end{center}
            \caption{The visible light filter in ToLuLu 1080p webcam
            and LifeCam VX-1000.}
            \label{fig:IR_cameras}
        \end{figure}

    \end{frame}

    \begin{frame}
        \frametitle{Software}
        \center
        Here we present the full list of our \textbf{\textcolor{blue}{Software}}
        used during experiment:

        ~\

        \begin{itemize}
            \item Eye tracking and calibration was provided by the
            \textbf{\textcolor{darkgray}{ITU\_Gaze\_Tracker}} codebase provided in
            class.

            \item This codebase was compiled in
            \textbf{\textcolor{darkgray}{Microsoft Visual Studio2017}}.

            \item IT was executed on a \textbf{\textcolor{darkgray}{Dell Inspiron 7386}}
            running \textbf{\textcolor{darkgray}{Windows 10}}.
        \end{itemize}

    \end{frame}

    \begin{frame}
        \frametitle{Creation of camera holder, chin rest, and assembly of the
        IR light for ToLuLu 1080p webcam}

        \begin{figure}
            \begin{center}
                \includegraphics[width=0.6\textwidth]{Work_space_Tolulu.jpg}
            \end{center}
            \caption{The workspace with \textcolor{brown}{ToLuLu 1080p webcam}.}
            \label{fig:Workspace_Tolulu}
        \end{figure}

    \end{frame}

    \begin{frame}
        \frametitle{Creation of camera holder, shin rest, and assembly of the
        IR light for LifeCam VX-1000}

        \begin{figure}
            \begin{center}
                \includegraphics[width=0.6\textwidth]{Work_space_VX_1000.jpg}
            \end{center}
            \caption{The workspace with \textcolor{brown}{LifeCam VX-1000}.}
            \label{fig:Workspace_VX}
        \end{figure}

    \end{frame}

    \begin{frame}
        \frametitle{Modification of the ITU Gaze Tracker}

        \begin{itemize}
            \item \texttt{\textcolor{blue}{CalibrationMenuUC.xaml.cs}} \\
                \begin{itemize}
                    \item Modified \texttt{\textcolor{darkgray}{SetAccuracy}}
                    method to display the calculated spatial precision value
                    alongside accuracy.
                \end{itemize}

            \item \texttt{\textcolor{blue}{CalibrationSettings.cs}}
                \begin{itemize}
                    \item Changed the value of
                    \texttt{\textcolor{darkgray}{defaultDistanceFromScreen}} to 600 mm.
                \end{itemize}

            \item \texttt{\textcolor{blue}{Interpolation.cs}}
                \begin{itemize}
                    \item Added accuracy calculation in
                    \texttt{\textcolor{darkgray}{CalculateDegreesLeft}} and
                    \texttt{\textcolor{darkgray}{CalculateDegreesRight}}.

                    \item Added a \texttt{\textcolor{darkgray}{degreesPrecision}}
                    variable and associated accessor methods.

                    \item Implemented the \texttt{\textcolor{darkgray}{CalculatePrecision}}
                    method to calculate spatial precision for the left eye.
                    The formula used to calculate spatial precision was adapted
                    from~\cite{Ho}.
                \end{itemize}

            \item \texttt{\textcolor{blue}{CalibrationWindow.xaml.cs}}
                \begin{itemize}
                    \item Modified \texttt{\textcolor{darkgray}{SetAccuracy}}
                    method call to include the spatial precision parameter.
                \end{itemize}

        \end{itemize}

    \end{frame}
    
    \begin{frame}
        \frametitle{The best result that we were able to get with LifeCam VX-1000}

        \center
        The best \textbf{spatial accuracy}: \ang{1.6} \\
        The best \textbf{spatial precision}: \ang{1.2}

        \begin{figure}
            \begin{center}
                \includegraphics[width=0.7\textwidth]{Best_res_VX_1000.jpg}
            \end{center}
            \caption{The best results in experiment with
            \textcolor{teal}{LifeCam VX-1000}.}
            \label{fig:Best_VX}
        \end{figure}

    \end{frame}

    \begin{frame}
        \frametitle{The best result that we were able to get with ToLuLu 1080p
        webcam}

        \center
        The best \textbf{spatial accuracy}: \ang{1.9} \\
        The best \textbf{spatial precision}: \ang{1.6}

        \begin{figure}
            \begin{center}
                \includegraphics[width=0.7\textwidth]{Best_res_Tolulu.jpg}
            \end{center}
            \caption{The best results in experiment with
            \textcolor{teal}{ToLuLu 1080p webcam}.}
            \label{fig:Best_Tolulu}
        \end{figure}

    \end{frame}

\end{document}